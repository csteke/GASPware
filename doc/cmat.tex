\documentstyle[11pt]{article}
\topmargin0cm
\oddsidemargin0cm
\evensidemargin0cm
\textwidth15cm
\textheight23cm
\parindent1cm
\begin{document}

\begin{center}
{\Large\bf\underline{CMAT PROGRAM}} \\
\bigskip
Part of the GASP Data Analysis Program Package\\
\bigskip
\bigskip
{D. Bazzacco} \\
\bigskip
{\small\it INFN, Sezione di Padova, Italy}\\
\bigskip
September 25, 1997
\end{center}

\bigskip
\bigskip
\bigskip

\noindent
{\large\bf 1. INTRODUCTION} \\

\bigskip
The {\bf CMAT} program is part of the Data Analysis Program Package developed
at Padova/Legnaro designed to analize multidimensional coincidence matrices
produced with GSORT program. 
It can execute the following operations:
\begin{enumerate}
\item 1D projections of the matrices;
\item extracting gated 1D spectra from 2D matrices;
\item extracting gated 2D matrices from 3D/4D matrices;
\item adding multidimensional matrices;
\item compress/decompress matrices;
\item transposing 2D matrices;
\item reading the content of one channel;
\item changing organization of a matrix;
\item 2D scatter plot of 2D matrices;
\item defining automatic banana gates on 2D matrices.
\end{enumerate}

When starting the program it shows the prompt {\bf CMAT$>$} and waits for a
command. Before acting on a matrix it has to be opened with the {\bf
OPEN} command. When the matrix is opened the program list its organization
(symmetric/nonsymmetric, dimensions of the axis, step values, number of
segments). A brief list of the commands can be get using the {\bf ?} command. 

\newpage
\noindent
{\large\bf 2. DETAILED DESCRIPTION OF THE COMMANDS} \\

\bigskip
\bigskip     

\begin{itemize}

 \item	{\it\underline{ADD}} \\

	It creates a matrix {\bf $fact3 \times mat3$} by summing the matrices 
	{\bf $fact1 \times mat1$} and {\bf $fact2 \times mat2$}. After the
	command the program ask for the name of the matrices to be summed, the 
	name of the final matrix and the factors to be mutiplied with; e.g.,
	with the factors 1 -1 1 one subtracts {\bf mat2} from {\bf mat1}. The
	two matrices involved in this operation must have the same
	organization.

 \item	{\it\underlin{BANANA\_DEFINE}} \\



 \item	{\it\underline{CHANNEL\_VALUE ~~ind1 ~ind2 ~...}} \\

	It shows the content and the address of a matrix location defined by
	${\bf ind1, ind2, ...}$.

 \item	{\it\underline{CLOSE}} \\

	Deallocates the matrix. This command allows to operate on a different
	matrix. Exiting from the program does not need explicit closure of the
	matrix.

 \item	{\it\underline{COMPRESS\_2D ~filename.ext}} \\

	It creates a compress matrix {\bf "filename.cmat"} starting from the
	decompressed one {\bf "filename.ext"} organized as {\bf res2} records
	of 2 bytes {\bf res1} channels. The program asks for the axes length
	and if they are equal it ask also if the compressed matrix should be
	symmetrized. When creating a symmetrized compressed matrix the content
	of each channel on the main diagonal is divided by 2 for consistency
	with the matrices created directly in compressed form.

 \item	{\it\underline{CUBEAID}} \\

	It is used to identify superdeformed structures with constant dynamical
	moment of inertia. It uses a 3D grid with constant spacing. There are 
	still problems with background subtraction.

 \item	{\it\underline{DECOMPRESS\_2D}} \\

	It produces a decompress matrix. Each location uses a 2 bytes integer
	and the file consists of {\bf res2} records of {\bf res1}$\times$ 2 
	bytes. This operation is valid only for 2D matrices. It allows to
	symmetrize the decompress matrix.

 \item	{\it\underline{DIAGN\_2D\_SYMM}} \\

	It creates a spectrum from a cut perpendicular on the main diagonal of
	a 2D symmetric matrix. It is used to evidence ridge structures. The 
	channels for the cut are selected from the projection of the matrix.
	The spectrum is created with a length equal to the projection af the
	matrix and its name is automatically generated including the channels
	in-between which the cut was done and the extension {\bf DIAGN}. Data
	are centered inside the spectrum. 

 \item	{\it\underline{DIAGP\_2D\_SYMM}} \\

	It creates a spectrum from a cut paralel to the main diagonal of
	a 2D symmetric matrix. The channels for the cut are selected from a 
	perpendicular cut on the main diagonal.
	The spectrum is created with a length equal to the projection af the
	matrix and its name is automatically generated including the channels
	in-between which the cut was done (distance in channels from the
	diagonal) and the extension {\bf DIAGP}.

 \item	{\it\underline{EXIT}} \\

	Exit the program. Open matrices are automatically closed.

 \item	{\it\underline{GATE}} \\

	It produces a 1D spectrum from the opened matrix setting gates on
	ndim-1 of its axis. Gates can be specified from a file or from terminal
	together with a normalization factor. The name of the spectrum is
	explicity asked together with its format.

 \item	{\it\underline{GET\_PROJECTIONS}} \\

	It reads the projections saved in the opened matrix file and writes
	them as spectra on disk. The name of the spectra are PROJ1.DAT,
	PROJ2.DAT, .... up to the number of axes (even in the case of
	symmetrized matrices).
	
 \item	{\it\underline{MOMENTS}} \\

	It calculates the area,center of mass and FWHM of a spectrum on one of
	the axes of a 2D matrix. Results are written as spectra and can be used
	for isomeric states identification from gamma-time matrices.

 \item	{\it\underline{M2D\_FROM\_M3D}} \\

	It produces a 2D matrix from a 3D matrix by setting a list of gates
	read from the terminal or from a disk file. One has to specify the axes
	which are projected. The time needed to execute this command depends on
	the matrix dimensions and the number of gates.

 \item	{\it\underline{OPEN ~filename}} \\

	It allocates the file containing the matrix {\bf filename.CMAT} (only
	{\bf READ} permission).

 \item	{\it\underline{PUT\_PROJECTIONS}} \\

	It writes the new projections spectra in a matrix opened with {\bf
	RWOPEN}.
		  
 \item	{\it\underline{RWOPEN ~filename}} \\

	It allocates the file containing the matrix {\bf filename.CMAT} 
	({\bf READ/WRITE} permission).


 \item	{\it\underline{SCATTERPLOT\_2D}} \\

	It allows to represent a 2D matrix as a scatter plot on a Tek4010
	graphical terminal. The first index is put on horizontal and the second
	one on vertical. It works only for nonsymmetric matrices. A few simple 
	commands in cursor-mode are available:

	{\bf T}~~~~~~~~~~~~~~~transposing the two indexes\\
	{\bf L}~~~~~~~~~~~~~~~left marker\\
	{\bf R}~~~~~~~~~~~~~~~right marker\\
	{\bf O}~~~~~~~~~~~~~~~upper marker\\
	{\bf U}~~~~~~~~~~~~~~~lower marker\\
	{\bf E}~~~~~~~~~~~~~~~expand the region defined by L R O U\\
	{\bf Z}~~~~~~~~~~~~~~~low offset on z-axis\\
	{\bf space\_bar}~~~shows the coordinates and content at the marker
				point\\
	{\bf CTRL\_Z}~~~~exit the graphics mode.\\

 \item	{\it\underline{SHIFT}} \\
	
	It changes the organizations of the opened matrix. It allows to pack
	together channles and to translate them (only by integer factors).

 \item	{\it\underline{STATISTICS}} \\
	
	It gives informations on the compression formatfor a given interval of
	segments.
 
 \item	{\it\underline{TEST}}\\
 
	It is used to test new commands under development.

 \item	{\it\underline{TRANSPOSE}} \\

	It allows to transpose a 2D matrix.

\end{itemize}

\end{document}
