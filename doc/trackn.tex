\documentstyle[11pt]{article}
\topmargin0cm
\oddsidemargin0cm
\evensidemargin0cm
\textwidth15cm
\textheight23cm
\parindent1cm
\begin{document}

\begin{center}
{\Large\bf\underline{TRACKN PROGRAM}} \\
\bigskip
Part of the GASP Data Analysis Program Package\\
\bigskip
\bigskip
{D. Bazzacco} \\
\bigskip
{\small\it INFN, Sezione di Padova, Italy}\\
\bigskip
September 25, 1997
\end{center}

\bigskip
\bigskip
\bigskip

\noindent
{\large\bf 1. INTRODUCTION} \\

\bigskip
The {\bf TRACKN} program is part of the Data Analysis Program Package developed
at Padova/Legnaro designed for graphical analisys of 2D coincidence matrices
produced with GSORT program and/or CMAT program.
It can execute the following operations:
\begin{enumerate}
\item displaying 1D spectra
\item extracting gated 1D spectra from 2D matrices with normal/common
	background subtraction;
\item simple operation with spectra (area, centroid and FWHM of peaks);
\item automatic peaksearch;
\item automatic calibration with standard gamma-ray sources;
\item defining smooth background spectrum and cut;
\item screen dump to a PS file.
\end{enumerate}

The program is running in graphic mode on Tek4010 terminals.
A brief list of the commands can be get using the {\bf ? or H} command. 

\newpage
\noindent
{\large\bf 2. DETAILED DESCRIPTION OF THE COMMANDS} \\

\bigskip
\bigskip     

\begin{itemize}

 \item	{\it\underline{SPACE\_BAR}} \\

	Set expand markers and display marker position in channels and energy
	if the calibration was defined before.

 \item	{\it\underline{E}} \\

	Expand spectrum between the last two 'space\_bar' markers.

 \item	{\it\underline{V}} \\

	It shows the position (in channels and energy) and spectrum content at
	the cursor position.

 \item	{\it\underline{SETTING MARKERS}} \\

	{\bf B}~~~~~~background marker (maximum four=two background region);\\
	{\bf I}~~~~~~integration marker (maximum two=one integration region);\\
	{\bf R}~~~~~~mark the ROI for fit (maximum two=one ROI);\\
	{\bf G}~~~~~~initial estimate of the peak position;\\
	{\bf W}~~~~~~gate marker;\\
	{\bf S}~~~~~~smooth background marker. 

 \item	{\it\underline{DELETING MARKERS}} \\

	{\bf ZB}~~~~~~cancel background markers;\\
	{\bf ZI}~~~~~~cancel integration marker;\\
	{\bf ZR}~~~~~~cancel ROI;\\
	{\bf ZG}~~~~~~cancel peak markers;\\
	{\bf ZW}~~~~~~cancel gate markers;\\
	{\bf ZS}~~~~~~cancel smooth background marker. 

 \item	{\it\underline{CB, CI, CJ, MI, MJ, AJ}} \\

	{\bf CB}~~~~~calculate/show background;\\
	{\bf CI}~~~~~integration between the I markers; area, centroid, FWHM
			listed\\
	{\bf CJ}~~~~~background+integration;\\
	{\bf MI}~~~~~show I markers;\\
	{\bf MJ}~~~~~show I and B markers;\\
	{\bf AJ}~~~~~automatic background+integration at cursor position.

 \item	{\it\underline{CG, CV, MG, MV, AG}} \\

	{\bf CG}~~~~~gaussian fit of the peaks marked with G inside the ROI; 
	             area, centroid, 
		     
		     \hskip1cm FWHM are listed;\\
	{\bf CV}~~~~~background+gaussian fit;\\
	{\bf MG}~~~~~show G markers;\\
	{\bf MV}~~~~~show G, R and B markers;\\
	{\bf AG}~~~~~automatic background+gaussian fit at cursor position.

 \item	{\it\underline{CP, DP, MP, ZP, +, -}} \\

	{\bf CP}~~~~~automatic peaksearch and show;\\
	{\bf DP}~~~~~define peaks from file;\\
	{\bf MP}~~~~~show peaks in the buffer;\\
	{\bf ZP}~~~~~cancel peak positions from buffer;\\
	{\bf ~+}~~~~~insert peak position at cursor location;\\
	{\bf ~-}~~~~~cancel peak position at cursor location.

 \item	{\it\underline{Dn, Cn, Mn, Zn, n}} \\

	{\bf Dn}~~~~~define command string (e.g., NFF=new spectrum+full display);\\
	{\bf Cn}~~~~~cicle on  the n-th command string;\\
	{\bf Mn}~~~~~show the n-th command string;\\
	{\bf Zn}~~~~~cancel the n-th command string;\\
	{\bf ~n}~~~~~execute the n-th command string (n=1,...,9).

 \item	{\it\underline{DS, MS, ZS}} \\

	{\bf DS}~~~~~read smooth background points from file;\\
	{\bf MS}~~~~~show smooth background points;\\
	{\bf ZS}~~~~~cancel smooth background points.

 \item	{\it\underline{DW, MW, ZW, CW, Q}} \\

	{\bf DW}~~~~~define cut limits from terminal or file;\\
	{\bf MW}~~~~~show cut limits;\\
	{\bf ZW}~~~~~delete cut limits;\\
	{\bf CW}~~~~~execute cut;\\
	{\bf ~Q}~~~~~return to the total matrix projection.

	{\bf CW} works in two ways:\\
	~~~~~~~~1) common backgound subtraction: all markers define peaks\\
	~~~~~~~~2) normal backgound subtraction: first two markers define the peak\\
	~~~~~~~~~~~~~~~~~~~~~~~~~~~~~~~~~~~~~~~~~all the others define the background
	
 \item	{\it\underline{DT, CT, AT}} \\

	{\bf DT}~~~~~;\\
	{\bf CT}~~~~~;\\
	{\bf AT}~~~~~.

 \item	{\it\underline{DD}} \\

	Define display characteristics. Some of the possibilities are:\\
	~~~~Y-display function: normal(default), logarithmic or square-root;\\
	~~~~X-limits (in channels);\\
	~~~~Y-limits (in channels);\\
	~~~~terminal window dimensions: x0, x1, y0, y1;\\
	~~~~text characteristics;\\
	~~~~ticks height;\\
	~~~~X-axis labels: calibrated(default) or in channels;\\
	~~~~X-axis title.

 \item	{\it\underline{DK, AK}} \\

	Define the energy and FWHM  calibrations.\\

	{\bf DK} allows to introduce the coefficients of the calibration 
	polynomials of up \\ 
	\hskip1cm to the fifth degree.

	{\bf AK} automatic energy calibration using a list of gamma-rays from a
	disk file, \\
	\hskip1cm introduced from terminal or with standard gamma-ray sources. 

 \item	{\it\underline{DL}} \\

	Define the LUN for the output of the calculations (fit, integration,
	...). If LUN=6 then th results are written on the top of the figure.


	
 \item	{\it\underline{DQ}} \\

	Define compressed matrix
	
 \item	{\it\underline{FF, FX, FY, FO, FU}} \\

	{\bf FF}~~~~~zoom off on both x and y axes;\\
	{\bf FX}~~~~~zoom off on the x axis; \\
	{\bf FY}~~~~~zoom off on the y axis; \\
	{\bf FO}~~~~~expand below the marker position on y axis ;\\
	{\bf FU}~~~~~expand above the marker position on y axis .

 \item	{\it\underline{L}} \\

	Turn off/on logaritmic scale

 \item	{\it\underline{N}} \\

	Define new spectrum 

 \item	{\it\underline{O=}} \\

	Screen dump of the Tektronix graphical window in a PS file or to 
	the print spooler

 \item	{\it\underline{P}} \\

	Go to a specified energy

 \item	{\it\underline{MZ}} \\


 \item	{\it\underline{$<$, $>$}} \\

	Shift spectrum to left/right by 3/4 of the x displayed range
		  
 \item	{\it\underline{=}} \\

	Redisplay


 \item	{\it\underline{CTRL\_C, CTRL\_Y, CTRL\_Z}} \\

	Exit (on UNIX use only CTRL\_Y)
	
\end{itemize}

\end{document}
